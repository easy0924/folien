%% LaTeX-Beamer template for KIT design
%% by Erik Burger, Christian Hammer
%% title picture by Klaus Krogmann
%%
%% version 2.1
%%uage
%% mostly compatible to KIT corporate design v2.0
%% http://intranet.kit.edu/gestaltungsrichtlinien.php
%%
%% Problems, bugs and comments to
%% burger@kit.edu

\documentclass[18pt]{beamer}

%% SLIDE FORMAT

% use 'beamerthemekit' for standard 4:3 ratio
% for widescreen slides (16:9), use 'beamerthemekitwide'

\usepackage{templates/beamerthemekit}
% \usepackage{templates/beamerthemekitwide}
\usepackage{csquotes} 
\usepackage{spreadtab}
\usepackage{multirow}
\usepackage{tikz}
\usetikzlibrary{arrows}
\usetikzlibrary{trees}
\usetikzlibrary{shapes}
\usetikzlibrary{chains}
\usetikzlibrary{calc}
\usetikzlibrary{positioning}

\usepackage[latin1]{inputenc} 
\usepackage[encapsulated]{CJK}
\newcommand{\cntext}[1]{\begin{CJK}{UTF8}{gbsn}#1\end{CJK}}

%% TITLE PICTURE

% if a custom picture is to be used on the title page, copy it into the 'logos'
% directory, in the line below, replace 'mypicture' with the 
% filename (without extension) and uncomment the following line
% (picture proportions: 63 : 20 for standard, 169 : 40 for wide
% *.eps format if you use latex+dvips+ps2pdf, 
% *.jpg/*.png/*.pdf if you use pdflatex)

%\titleimage{mypicture}

%% TITLE LOGO

% for a custom logo on the front page, copy your file into the 'logos'
% directory, insert the filename in the line below and uncomment it

%\titlelogo{mylogo}

% (*.eps format if you use latex+dvips+ps2pdf,
% *.jpg/*.png/*.pdf if you use pdflatex)

%% TikZ INTEGRATION

% use these packages for PCM symbols and UML classes
% \usepackage{templates/tikzkit}
% \usepackage{templates/tikzuml}

% the presentation starts here

%\title[Short title]{Analyzing the Potential of Source Sentence Reordering in SMT for Chinese}
%\subtitle{Multi-Level-Tree Rule Based Preordering Approach}

\title{Chinese Preordering on Multiple Syntactic Levels}
\subtitle{}
\author{Ge Wu}

\institute{Institute for Anthropomatics and Robotics (IAR)}

% Bibliography

\usepackage[citestyle=authoryear,bibstyle=numeric,hyperref,backend=bibtex]{biblatex}
\addbibresource{folien.bib}
\bibhang1em

\begin{document}

% change the following line to "ngerman" for German style date and logos
\selectlanguage{english}

%title page
\begin{frame}
\titlepage
\end{frame}

%table of contents
\begin{frame}{Outline}
\tableofcontents
\end{frame}

\section{Introduction}
\begin{frame}{Introduction}
Goal \& Motivation \& Reason, etc.
\begin{itemize}
\item X
\end{itemize}
\end{frame}

\section{Foundations}

%\subsection{System}
\begin{frame}{Preordering System}
\begin{figure}
\centering
\newlength\mylens
\setlength{\mylens}{1cm}

\begin{tikzpicture}[scale=0.7,
->,>=stealth', grow=right, level 1/.style={sibling distance=1.3\mylens}, level distance=4\mylens,
node/.style = {scale=0.7, align=center, inner sep=0pt, text centered, font=\sffamily, rectangle, rounded corners, draw=black, thick, fill=blue!20, text width=5em, minimum height = 2em, inner sep=5},
nodeimp/.style = {node, fill=red!20},
lab/.style={scale=0.7}
]


\node(A) [node, text width=8em] at (0, 0) {Source Sentences};
%\node(B) [node, below=\mylens of A] {Reordering};
\node(B) [draw=black, thick, circle, below=\mylens of A] {};
\node(C) [node, text width=8em, text height=4ex, below=2\mylens of B] {Decoder\\ \vphantom{x}};

\node (CW) [left=1pt of C.south west] {};
\node (CE) [right=1pt of C.south east] {};
%\draw[-, line width=10pt, white] (CW) to (CE);
%\node(XX) [below=0.1\mylens of C] {};
%\node(X) [node, draw=white, rounded corners=0, fill=white, maximum height = 0.1em] at (C.south) {};

\node(E) [nodeimp, right=1.4\mylens of B] {Reordering Rules};
\node(EE) [draw=black, thick, circle, right=1.6\mylens of E] {};

\node(F) [nodeimp, above right=0.3*\mylens and 2.85\mylens of E] {Word Alignment};
\node(G) [nodeimp, right=2.85\mylens of E] {POS Tags};
\node(H) [nodeimp, below right=0.3*\mylens and 2.85\mylens of E] {Syntactic Tree};

\node(I) [node, right=\mylens of G] {Training Data};


\draw[->, thick] (A) to (B);
\draw[white] (C) to node[lab, black, midway, sloped, above] {Lattices} (B);
\draw[->, thick] (B) to (C);
\draw[->, thick] (E) to node[lab, midway, above] {Apply} (B);
\draw[->, thick] (EE) to node[lab, midway, above] {Extract} (E);

\node(Saa) [right=0.5\mylens of EE] {};
\node(Sbb) [left=0.5\mylens of I] {};

\coordinate(Sa) at (Saa.base);
\coordinate(Sb) at (Sbb.base);

\draw[->, thick] (Sa) to (EE);
\draw[-, thick] (I) to (Sb);

\draw[-, thick] (F.west) -| (Sa);
\draw[-, thick] (G.west) -| (Sa);
\draw[-, thick] (H.west) -| (Sa);

\draw[->, thick] (Sb) |- (F.east);
\draw[->, thick] (Sb) |- (G.east);
\draw[->, thick] (Sb) |- (H.east);

%\node(2) [below=1cm of A, node, minimum height = 10 em] at (0\myxa,3\myya) {Decoder};
%\node(3) [node] at (0\myxa,2\myya) {Target Sentences};


%\node(1) [nodeimp] at (3\myxa,4\myya) {Reordering Rules};

%\draw[->] (0) to node [midway, sloped, below] {} node [midway, sloped, above] {} (1);

\end{tikzpicture}
%\caption{Preordering system}
%\label{prereordering}
\end{figure}
\end{frame}


%\subsection{Reordering Rules}
\begin{frame}{Rreordering Rules}
\begin{itemize}
\item Short rules
\item Long rules
\item Tree rules
\end{itemize}
\end{frame}


%\subsection{Chinese Word Orders}
\begin{frame}{Chinese Word Orders}
\begin{itemize}
\item Premodifier instead of postmodifier
\begin{itemize}
	\item Adverbials
	\item Relative clauses
	\item Preposition phrases
\end{itemize}	
\end{itemize}
\begin{figure}
\centering
\begin{tikzpicture}[
node/.style = {
text centered, 
text height=1.5ex,
text depth=.25ex,
inner sep=2pt, font=\sffamily, rectangle, draw=none, fill=none, outer sep=0,
minimum height=4ex
},
node2/.style = {
text height=4.25ex, text depth=.25ex, draw=black, inner sep=0, outer sep=0, rounded corners
},
]

\node(A1) [node] at (0, 2.5) {Those};
\node(A2) [node, right=1em of A1] {are};
\node(A3) [node, right=1em of A2] {conveyor};
\node(A4) [node, right=1em of A3] {belts};
\node(Ax) [node2, right=1em of A4] {
\tikz\node(A5) [node] {that};
\tikz\node(A6) [node, right=1em of A5] {go};
\tikz\node(A7) [node, right=1em of A6] {around};
};
\node(A8) [node, right=1em of Ax] {.};

\node(B1) [node] at (0, 0) {\cntext{那些}};
\node(B2) [node, right=2.475em of B1] {\cntext{是}};
\node(Bx) [node2, right=2.475em of B2] {
\tikz\node(B3) [node] {\cntext{在}};
\tikz\node(B4) [node, right=2.475em of B3] {\cntext{运转}};
\tikz\node(B5) [node, right=2.475em of B4] {\cntext{的}};
};
\node(B6) [node, right=2.475em of Bx] {\cntext{传送带}};
\node(B7) [node, right=2.475em of B6] {\cntext{。}};

%1-1 2-2 3-6 4-6 5-6 6-6 7-6 8-7

\draw[dashed] (A1.south) -- (B1.north);
\draw[dashed] (A2.south) -- (B2.north);
\draw[dashed] (Ax.south) -- (Bx.north);
\draw[dashed] (A3.south) -- (B6.north);
\draw[dashed] (A4.south) -- (B6.north);
\draw[dashed] (A8.south) -- (A8|-B7.north);
\end{tikzpicture}

\end{figure}
\end{frame}

\begin{frame}{Chinese Word Orders}
\begin{itemize}
\item Questions
\item Special sentence constructions
\begin{itemize}
\item \textit{\alert{There aren't} many people around that are really involved with architecture as clients.}
\item \textit{\alert{Never would} India have thought on this scale before.}
\end{itemize}
\end{itemize}
\end{frame}

\begin{frame}{Reordering Problems}
\begin{itemize}
\item Long distance position change
\item Reordering on multiple syntactic levels
\end{itemize}
\end{frame}


\section{Reordering Approach}
\begin{frame}{Reordering Approach}
Rule Extraction \& Application
\begin{itemize}
\item X
\end{itemize}
\end{frame}


\section{Results}
\begin{frame}{Results}
\begin{table}
\centering
\STautoround*{2}
\begin{spreadtab}{{tabular}{|l|r|r|}}\hline
@				& @BLEU Score & @Improvement \\ \hline
@Baseline		& 12.07 & \\ \hline
@+Short Rules	& 12.50 & :={b3 * 100 /b2 - 100} \% \\ \hline
@+Long Rules   & 12.99 & :={b4 * 100 /b2 - 100} \% \\ \hline
@+Tree Rules   & 13.38 & :={b5 * 100 /b2 - 100} \% \\ \hline
@+MLT Rules    & 13.81 & :={b6 * 100 /b2 - 100} \% \\ \hline
@Oracle Reordering & 18.58 & :={b7 * 100 /b2 - 100} \% \\ \hline
\hline
@Long Rules   & 12.31 & :={b8 * 100 /b2 - 100} \% \\ \hline
@Tree Rules   & 13.30 & :={b9 * 100 /b2 - 100} \% \\ \hline
@MLT Rules    & 13.68 & :={b10 * 100 /b2 - 100} \% \\ \hline
\end{spreadtab}
\caption{BLEU score overview of English-to-Chinese system}
\label{tenw}
\end{table}
\end{frame}

\begin{frame}{Results}
\begin{table}
\centering
\STautoround*{2}
\begin{spreadtab}{{tabular}{|l|r|r|}}\hline
@				& @BLEU Score & @Improvement \\ \hline
@Baseline		& 21.80 & \\ \hline
@+Short Rules	& 22.90 & :={b3 * 100 /b2 - 100} \% \\ \hline
@+Long Rules   & 23.13 & :={b4 * 100 /b2 - 100} \% \\ \hline
@+Tree Rules   & 23.84 & :={b5 * 100 /b2 - 100} \% \\ \hline
@+MLT Rules    & 24.14 & :={b6 * 100 /b2 - 100} \% \\ \hline
@Oracle Reordering & 26.80 & :={b7 * 100 /b2 - 100} \% \\ \hline
\hline
@Long Rules   & 22.10 & :={b8 * 100 /b2 - 100} \% \\ \hline
@Tree Rules   & 23.35 & :={b9 * 100 /b2 - 100} \% \\ \hline
@MLT Rules    & 23.96 & :={b10 * 100 /b2 - 100} \% \\ \hline
\end{spreadtab}
\caption{BLEU score overview of Chinese to English systems}
\label{tzhen2}
\end{table}
\end{frame}



\section{Conclusion}
\begin{frame}{Conclusion}
\begin{itemize}
\item Better translation quality
\item Better syntactic structure
\item Space for further improvement
\end{itemize}
\end{frame}


\section{Outlook}
\begin{frame}{Outlook}
\begin{itemize}
\item Other rule types
\item Better reordering approaches
\item Vector presentation as feature
\item Reordering with less information
\end{itemize}
\end{frame}


\begin{frame}{\vphantom{A}}
\centering
{\LARGE \textbf{Thank you for your attention}}
\end{frame}


%\section{Section 1}
%\subsection{Subsection 1.1}
%\begin{frame}{Example slide A}
%\begin{itemize}
%\item PCM, Citation: \cite{long}
%\pause
%\item Bullet point 2
%\item \dots
%\end{itemize}
%\end{frame}
%
%\subsection{Subsection 1.2}
%\begin{frame}{Example slide B}
%\begin{block}{Block 1}
%\begin{itemize}
%\item Bullet point 1
%\pause
%\item Bullet point 2
%\item \dots
%\end{itemize}
%\end{block}
%\end{frame}
%
%\section{Section 2}
%\begin{frame}{Example slide C}
%\begin{exampleblock}{Example 1}
%\begin{itemize}
%\item Bullet point 1
%\pause
%\item Bullet point 2
%\item \dots
%\end{itemize}
%\end{exampleblock}
%\end{frame}
%
%\begin{frame}{Example slide D}
%\begin{alertblock}{Alert 1}
%\begin{itemize}
%\item Bullet point 1
%\pause
%\item Bullet point 2
%\item \dots
%\end{itemize}
%\end{alertblock}
%\end{frame}

\appendix
\beginbackup

\nocite{*}
\begin{frame}[allowframebreaks]{References}
\printbibliography
\end{frame}

\backupend

\end{document}
